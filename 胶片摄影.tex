\documentclass[a4paper,12pt]{ctexart}
\usepackage{amsmath, amsthm, amssymb, bm, graphicx, float, hyperref, mathrsfs}
\title{{\Huge{\textbf{胶片摄影}}}\\\textbf{Film Photography}}
\author{李佳齐}
\date{2022年5月23日}
\linespread{1.5}
\begin{document}
\maketitle
\clearpage
\tableofcontents
\clearpage
\part{相机}

我的相机是国产品牌,Phenix DC828M,来自凤凰光学股份有限公司。下面这张图就展示了我的相机机身,\textit{以及我钟爱的好丽友·派}。
\begin{figure}[H]
	\centering
	\begin{minipage}[t]{0.48\textwidth}
		\centering
		\includegraphics[width=6cm]{figures/相机正视图.jpg}
		\caption{我的Phenix DC828M}
	\end{minipage}
	\begin{minipage}[t]{0.48\textwidth}
		\centering
		\includegraphics[width=6cm]{figures/相机俯视图.jpg}
		\caption{别开我的后盖}
	\end{minipage}
\end{figure}
\par 为什么买胶片相机?一是想要通过拍胶卷来体会什么叫做“胶片感”,往小了说在摄影后期时会有帮助,往大了说能帮助我构建一个更广阔的、基于实践的风格空间或者审美空间;二是我本人对古典之美尤为欣赏,胶片相机的机械构造,胶卷的成像原理,甚至是胶片相机的外形,过片的手感,慢拍摄的体验,这些都吸引着我;三是出于实际的考虑,胶片相机便宜,虽然胶卷比较贵,但是相较于胶片模拟十分出色的富士微单而言,还是便宜了很多。
\par 我不仅对相机这种机械感十足的玩意兴趣浓厚,而且喜欢摄影。摄影不仅是记录,也是表达,如同文字的使命。接下来我将尽我所能,基于实践、阅读和交流,介绍相关的知识技巧。

\section{光路}

所谓“单反相机”,全称单镜头反光式取景相机(Single Lens Reflex Camera, SLR),这是一种光路的设计。与之并列的,还有双反、旁轴、微单等。
\begin{figure}[H]
	\centering
	\begin{minipage}[t]{0.48\textwidth}
		\centering
		\includegraphics[width=6cm]{figures/相机结构示意.png}
		\caption{单反构造}
	\end{minipage}
	\begin{minipage}[t]{0.48\textwidth}
		\centering
		\includegraphics[width=6cm]{figures/旁轴相机光路.jpg}
		\caption{旁轴相机光路}
	\end{minipage}
\end{figure}
\par 用单镜头,使光线通过此镜头投射至对焦屏上,通过反光取景。这种构造使得单反相机不能太小,因为你要在相机里面装下一个五棱镜。与之相对应,旁轴相机没有这个限制,所以它的体型更小一点。
\par 单反相机头上都有个凸起,里面装的是五棱镜。它并不是五棱柱或者五棱锥,它有9个面,只是有一个面是五边形,但这不重要。五棱镜的作用是让原本左右颠倒的图像给正过来。如图5所示,光线从下面这个面传入,经过多次反射,到左边这个侧面传出——\textit{这时候你要赶紧用眼睛去接住,要不画面就会以光速逃逸}。
\begin{figure}[H]
	\centering
	\begin{minipage}[t]{0.48\textwidth}
		\centering
		\includegraphics[width=6cm]{figures/五棱镜光路.png}
		\caption{五棱镜光路}
	\end{minipage}
	\begin{minipage}[t]{0.48\textwidth}
	\centering
	\includegraphics[width=6cm]{figures/五棱镜.jpg}
	\caption{五棱镜}
\end{minipage}
\end{figure}

\section{装胶卷}

刚买回来的时候,不仅不知道胶卷暗盒的构造,而且不知道怎么装胶卷,怎么知道装没装好。这是对于胶片新手而言,最紧迫的、\textit{生死攸关的一个问题,解决不了就会死},所以我来帮忙解答一下。
\par 胶卷暗盒由一个轴、两块外壳组成。轴像是健身房的哑铃:中间细,缠绕胶卷;两头宽,构成暗盒这个圆柱体的两个底面。上底面能与相机的倒片轴咬合,倒片轴是相机的一个元件,一般在相机的左侧,这个元件在外面看是一个圆钮(图2右侧),把相机后盖打开了从里面看,它伸到了暗盒室里面(图11左侧),是倒片旋钮在暗盒室的那部分。两个外壳组成口哨型结构,暗盒口处有两层海绵,可能是兼具防尘和避光的作用,\textit{但我觉得这海绵没法用来洗碗,不喜勿喷}。
\begin{figure}[H]
	\centering
	\begin{minipage}[t]{0.48\textwidth}
		\centering
		\includegraphics[width=6cm]{figures/暗盒俯视图.jpg}
		\caption{胶卷暗盒俯视图}
	\end{minipage}
	\begin{minipage}[t]{0.48\textwidth}
		\centering
		\includegraphics[width=6cm]{figures/暗盒结构.jpg}
		\caption{胶卷暗盒结构}
	\end{minipage}
\end{figure}
\par 装胶卷时,遵循以下步骤
\begin{itemize}
	\item 先把胶卷头插入卷片轴,胶片两侧边缘的齿孔要咬合到卷片轴左边的输片轴上的齿;
	\item 然后手动给卷片轴推转半圈,推转的时候再次确保胶片的齿孔与输片齿轮咬合;
	\item 之后固定胶卷片头拉动胶卷暗盒,将其放进相机的暗盒室,注意暗盒凸起的那个底面要朝下;
	\item 最后盖上后盖,按下倒片旋钮,它会与胶卷暗盒相咬合。
\end{itemize}
\par 扳动卷片扳手,注意观察卷片轴能通过胶卷的连接带着倒片旋钮一起转,如果过片的时候倒片旋钮能跟着旋转,那就说明装好胶卷了,\textit{这让咱们胶片新手激动地直跺脚}。
\begin{figure}[H]
	\centering
	\begin{minipage}[t]{0.48\textwidth}
		\centering
		\includegraphics[width=6cm]{figures/胶卷齿孔.jpg}
		\caption{胶卷的齿孔}
	\end{minipage}
	\begin{minipage}[t]{0.48\textwidth}
		\centering
		\includegraphics[width=6cm]{figures/装胶卷.png}
		\caption{正确装胶卷}
	\end{minipage}
\end{figure}
\par 尤其要注意,不建议先装暗盒再拉片头。把片头插进卷片轴是需要操作空间的,你要么把胶卷拉出来很长,要么就很难插进卷片轴。我说的安装胶卷方法更快、更牢靠,也能让你多拍一两张。毕竟胶卷贵,\textit{咱们还是秉持着骑自行车去酒吧的精神,该省省该花花}。
\par 装完胶卷,要调整相机上的ISO拨盘,与你的胶卷一致,否则会给测光带来系统误差。

\section{取胶卷}

当你拍完了一卷,取胶卷之前,要先倒片,也就是把胶卷重新倒回暗盒。这里需要先解释一下过片时发生了什么。卷片轴与卷片扳手是咬合在一起的,过片时你扳动卷片扳手,卷片轴会相反地沿顺时针旋转,让胶片缠绕上去。胶片的向右移动会带着输片齿轮逆时针旋转,而输片齿轮也刚好被设计为只能逆时针转——\textit{此中有真意,欲辨已忘言}。同时呢,胶片另一端连着暗盒的轴,会让暗盒的轴连带着与之咬合的倒片轴、倒片旋钮逆时针旋转,而暗室的外壳保持不动。这样,当你拍完一卷后,胶卷就全从左边的暗盒转移到右边的卷片轴了。
\begin{figure}[H]
	\centering
	\begin{minipage}[t]{0.48\textwidth}
		\centering
		\includegraphics[width=6cm]{figures/相机后背.png}
		\caption{相机后背}
	\end{minipage}
	\begin{minipage}[t]{0.48\textwidth}
		\centering
		\includegraphics[width=6cm]{figures/过片与倒片.jpg}
		\caption{过片与倒片}
	\end{minipage}
\end{figure}
\par 倒片时,先按相机底部的倒片按钮,这样输片齿轮是才能顺时针转动!才能让胶片往左运动。然后扳起倒片手柄,顺时针旋转,把胶片卷回去。如果感到阻力明显减小,能听到哒一声,胶卷就全部脱离了卷片轴和输片轴,它哥俩都不再旋转了,这时候已经可以停止了;如果再旋转,摩擦力会更小,因为胶片就被全部卷进暗盒了,它不再与暗盒口的海绵摩擦。
\par 咱们如果只是送出去冲洗,怎么着都行;但是如果是自己冲洗,最好还是听到哒一声就停下右手的动作,留一段片头在暗盒外头,不然还得用个抽片器把片头抽出来。

\clearpage
\part{胶卷}
胶片结构主要是感光乳剂和片基。片基好说,一般是乙酸脂薄膜,有韧性、透明、惰性、不易燃。感光乳剂稍微复杂一点,结合感光过程讲一下:乳剂由悬浮于明胶介质的卤化银组成,卤化银晶体遇光改变结构,在胶片上形成潜影——这就是拍摄的全部,后面的就属于冲洗了。经过显影,潜影转化为黑色银颗粒聚集而成的负像;经过定影,未感光的晶体被洗去而在胶片上透明。
\par 评价胶片的特性,一般看这几个方面:感光速度,颗粒度,反差。接下来,我结合我的实践,依次介绍几类胶片的使用体验。
\section{黑白胶片}
拍黑白胶卷,从学习上讲是一种捷径,因为你只需要关注构图和影调,不需要关心颜色,就能达到很好的效果。如果你看\textit{Roman Holid}y,\textit{Gone with the Wind}这些老黑白电影,你能欣赏\textit{Better Call Saul}的黑白片段,你应该就不会揶揄拍黑白的人是在拍遗照。有相当一部分人十分推崇黑白摄像。
\par 黑白负片我用过伊尔福(ILFORD PAN100, ILFORD HP5 PLUS)和福马(FOMA PAN100)。体验下来,宽容度全都够用,福马颗粒度大,影调硬朗;伊尔福颗粒度小,影调十分柔和。我用ILFORD PAN100在圆明园拍出过非常满意的照片。
\begin{table}[H]\centering
	\begin{tabular}{cccccc}
		\hline\hline
		胶片型号 & 类型 & 感光度 & 宽容度 & 颗粒度 & 对比度 \\
		\hline
		Foma pan100 & 黑白负片 & 100 & [-2,+1] & 中等偏细 & 很高 \\
		Ilford pan100 & 黑白负片 & 100 & [-2,+2] & 中等 & 中等 \\
		Ilford hp5 plus & 黑白负片 & 400 & [-3,+4] & 中等,经典 & 中等 \\
		Kodak 5207 & 彩色电影负片 & 250 & [-2.5,+5] & 极细 & 很低 \\
		Kodak 5219 & 彩色电影负片 & 500 & [-3,+5] & 很细 & 很低 \\
		Kodak 5294 & 彩色反转片 & 100 & [-1.5,+1] & 极细 & 极高 \\
		\hline\hline
	\end{tabular}
\end{table}
\section{彩色胶片}
关于彩色胶片,富士色彩科学的地位是得到摄影圈公认的,甚至到数码相机时代,还会有好多人因为富士的胶片模拟而选择富士产品。我目前用的柯达5219和5207,都是电影卷。拍摄过程中发现,这些胶卷在阴影的细节保留较少,但在高光里保留细节较多,因而尽量避免欠曝,不然在后期过程中阴影会有很多噪点。
\par 彩色反转片我用过柯达5294,宽容度很低,适合小光比场景。
\clearpage
\part{摄影}
\section{曝光与测光}
作为一名理科生,我需要现厘清物理光强和感知亮度的关系。根据韦伯-费希纳定律,在一定范围内,人眼对亮度的感知与物理光强之间是对数关系,所以我们总是把光强提高2倍(物理光强)叫做“提高一档”(感知亮度)。
\begin{equation*}
	y=a+b\log_{2}x \quad\Rightarrow\quad \Delta y\sim\Delta\log_{2}x
\end{equation*}
\par 接下来正式讲曝光,从中性灰开始。中性灰也叫18度灰,本质上是对光线18\%的反射率。人们认为,大量场景的平均反射率是18\%,于是将其定为测光表的基准。可以认为这是咱们现代曝光体系的公理,它规定了测光的标准。\textit{去接受它、感受它,不要尝试质疑它,否则会变得不幸}。接收了中性灰的公理后,我们发现相机的测光表所做的事情只是,读取选定区域的反射光,然后告诉我们快门和光圈的哪些组合能够带来18\%的灰色影调。
\par \textit{如果说中性灰只是曝光数轴上的一个点,接下来我们就要讲区间了。}场景有光强范围的分布,我们将胶片、相机能记录的光强范围称为宽容度。肉眼的宽容度大约是50000,即人眼能分辨出细节的最大光强与最小光强之比为50000;胶片的宽容度则低很多,比如HP5 plus的宽容度约为125。
\par 在摄影实践中,我们一般要结合曝光的档位重新叙述宽容度。比如提高7档曝光对应光强提高128倍,所以摄影实践中一般说HP5 plus有7档宽容度。然后需要说的是,也是我体验很深的事情,胶片的宽容度差别较大且不对称,彩色负片在阴影的宽容度较窄,容易让画面的阴影出现损失细节的死黑;彩色反转片宽容度尤其低且很容易过曝。
\par 考虑到胶片对高光和阴影的宽容度不对称,一般遵循如下原则:黑白胶片或者彩负,对阴影进行曝光;反转片,对高光进行曝光。比如对画面中的阴影测光,选定ISO和快门时间后发现光圈值f2能得到中性灰,那么对于黑白胶片,可以收缩两档光圈,按照f4拍摄,这样能保证成片的阴影不会欠曝。
那么反转片要对高光进行曝光,显然是因为反转片对高光的宽容度更低。具体操作类推即可。

\section{测光经验法则}
在摄影中,曝光值$EV$与感光度$ISO$,快门$T$,光圈数$N$(即$f$值)的关系为
\[EV=\log_{2}\frac{N^{2}}{T\cdot ISO/100}\]
如果曝光与测光的关系太过复杂,无妨,咱们有“阳光十六法则”,即
\begin{table}[H]\centering
	\textit{阳光十六阴天八}\\
	\textit{多云十一日暮四}\\
	\textit{乌云压顶五点六}\\
	\textit{雪天落雨同日暮}
\end{table}
\noindent 就能应付大多数自然场景的测光需求了。当然如果有更细致的场景区分和测光需求,可以参考下表
\par\begin{center}\begin{tabular}{ll}
		\hline
		16 阳光明媚&\ 5 夜间家居室内\\
		15 阳光明媚,有轻微雾霾&\ 4 泛光照明的建筑、蜡烛灯\\
		14 有阳光但稍微阴&\ 3 烟花\\
		13 多云的明亮光线&\ 2 闪电\\
		12 有明显阴影&\ 1 遥远的天际线\\
		11 多云&\ 0 昏暗的人造光\\
		10 日落时的风景&-1 昏暗的人造光\\
		\ 9 日落后的风景&-2 夜晚,远离城市,满月下雪景\\
		\ 8 时代广场&-3 晚上,远离城市,满月下\\
		\ 7 室内运动场、明亮的夜间街道&-4 晚上,远离城市,半月下\\
		\ 6 明亮的室内、展览会、游乐园&-5 晚上,远离城市,新月下\\
		\hline
\end{tabular}\end{center}


\end{document}